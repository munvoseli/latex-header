\documentclass{article}
\usepackage{geometry}
\geometry{a4paper, portrait, margin=2cm}
\usepackage{amsmath}
\usepackage{mathtools}
\usepackage{graphicx}
\graphicspath{ {./} }
\usepackage[framemethod=TikZ]{mdframed}
\usepackage{lipsum}

\definecolor{boxbg}{HTML}{FFFEB7}
\definecolor{bonkle}{HTML}{FF1500}

\DeclareMathOperator{\sgn}{sign}
\DeclarePairedDelimiter\abs{\lvert}{\rvert}
\DeclarePairedDelimiter\magn{\lvert\lvert}{\rvert\rvert}
\DeclarePairedDelimiter\ceil{\lceil}{\rceil}
\DeclarePairedDelimiter\floor{\lfloor}{\rfloor}

\DeclareRobustCommand{\eel}[1]{\begin{align*}{#1}\end{align*}}
\newmdenv[roundcorner=8pt,align=center,linecolor=boxbg,
  backgroundcolor=boxbg,linewidth=4,outerlinewidth=2,outerlinecolor=bonkle]{stonk}

\begin{document}

\begin{align*}
43x^2+27x-5&=0\\
-43*20x^2-2*27*2*5+5*5*2*2&=0\\
-43*20x^2-2*27*10+10^2&=0\\
27^2x^2-2*27*10x+10^2&=(27^2+43*20)x^2\\
(27x-10)^2&=(27^2+43*20)x^2\\
(27-10/x)^2&=27^2+43*20\\
27-\frac{10}x&=\pm\sqrt{27^2+43*20}\\
27\pm\sqrt{27^2+43*20}&=\frac{10}x\\
\frac{10}{27\pm\sqrt{27^2+43*20}}&=x
\end{align*}

And now, some conjugate magic.  It works out if you do both + and - in separate paths tho

\begin{align*}
x&=\frac{10(27\pm\sqrt{27^2+43*20})}{27^2-(27^2+43*20)}\\
&=\frac{10(27\pm\sqrt{27^2+43*20})}{-43*20}\\
&=\frac{-10(27\pm\sqrt{27^2+43*20})}{43*20}\\
&=\frac{-(27\pm\sqrt{27^2+43*20})}{43*2}\\
&=\frac{-27\pm\sqrt{27^2+43*20}}{43*2}\\
&=\frac{-27\pm\sqrt{680+49+860}}{86}\\
&=\frac{-27\pm\sqrt{729+860}}{86}\\
x&=\frac{-27\pm\sqrt{1589}}{86}\\
\end{align*}

\begin{stonk}
hi
\end{stonk}

\end{document}
