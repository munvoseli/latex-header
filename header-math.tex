\documentclass{article}
\usepackage{geometry}
\geometry{a4paper, portrait, margin=2cm}
\usepackage{amsmath}
\usepackage{mathtools}
\usepackage{graphicx}
\graphicspath{ {./} }
\usepackage[framemethod=TikZ]{mdframed}
\usepackage{lipsum}

\definecolor{boxbg}{HTML}{FFFEB7}
\definecolor{bonkle}{HTML}{FF1500}

\DeclareMathOperator{\sgn}{sign}
\DeclarePairedDelimiter\abs{\lvert}{\rvert}
\DeclarePairedDelimiter\magn{\lvert\lvert}{\rvert\rvert}
\DeclarePairedDelimiter\ceil{\lceil}{\rceil}
\DeclarePairedDelimiter\floor{\lfloor}{\rfloor}
\DeclarePairedDelimiter\vecnum{\langle}{\rangle}

\DeclareRobustCommand{\eel}[1]{\begin{align*}{#1}\end{align*}}
\newmdenv[roundcorner=8pt,align=center,linecolor=boxbg,
  backgroundcolor=boxbg,linewidth=4,outerlinewidth=2,outerlinecolor=bonkle]{stonk}

\begin{document}

On Debian, in addition to the normal recommended package, you'll probably also have to install things like texlive-latex-extra or texlive-science to get all the necessessary things.

\begin{align*}
  \vec a=\vecnum{0,5,3}\\
  \sgn 5+\abs{3-5}\\
  \magn{\vecnum{2,5,14}}=15\\
  \floor{5.5}=\ceil{5}=5\\
  \ceil{5.5}=\floor{6.7}=6
\end{align*}

\begin{align*}
43x^2+27x-5&=0\\
-43*20x^2-2*27*2*5+5*5*2*2&=0\\
-43*20x^2-2*27*10+10^2&=0\\
27^2x^2-2*27*10x+10^2&=(27^2+43*20)x^2\\
(27x-10)^2&=(27^2+43*20)x^2\\
(27-10/x)^2&=27^2+43*20\\
27-\frac{10}x&=\pm\sqrt{27^2+43*20}\\
27\pm\sqrt{27^2+43*20}&=\frac{10}x\\
\frac{10}{27\pm\sqrt{27^2+43*20}}&=x
\end{align*}

And now, some conjugate magic.  It works out if you do both + and - in separate paths tho

\begin{align*}
x&=\frac{10(27\pm\sqrt{27^2+43*20})}{27^2-(27^2+43*20)}\\
&=\frac{10(27\pm\sqrt{27^2+43*20})}{-43*20}\\
&=\frac{-10(27\pm\sqrt{27^2+43*20})}{43*20}\\
&=\frac{-(27\pm\sqrt{27^2+43*20})}{43*2}\\
&=\frac{-27\pm\sqrt{27^2+43*20}}{43*2}\\
&=\frac{-27\pm\sqrt{680+49+860}}{86}\\
&=\frac{-27\pm\sqrt{729+860}}{86}\\
x&=\frac{-27\pm\sqrt{1589}}{86}\\
\end{align*}

\begin{stonk}
hi
\end{stonk}

\section*{Vector product things}

section 12.4, table 11

Cross product produces a vector that is perpendicular to the input vectors.  Magnitude of the cross product is the area of the parallelogram made by the two vectors.

\begin{stonk}
  Cross product works only with 3 dimensional vectors (for now, at least?).
  
  Important equations:
  \begin{align*}
    \magn{\vec a}^2=\vec a\cdot\vec a\\
    \vec{a}\cdot\vec{b}=\magn{\vec{a}}*\magn{\vec{b}}*\cos\theta\\
    \magn{\vec{a}\times\vec{b}}=\magn{\vec{a}}*\magn{\vec{b}}*\sin\theta\\
    A_{2-parallelogram}=\magn{\vec a\times\vec b}\\
    A_{3-parallelogram}=(\vec a\times\vec b)\cdot\vec c\\
    \vec a\perp\vec a\times\vec b\perp\vec b
  \end{align*}
  Fun facts as a result:
  \begin{align*}
    \vec{a}\cdot\vec{b}=0&\Leftrightarrow\vec a\perp\vec b\\
    \magn{\vec{a}\times\vec{b}}=0&\Leftrightarrow\vec a\parallel\vec b\\
  \end{align*}
\end{stonk}


\subsection*{Volume of 3-parallelogram}

\begin{align*}
  V&=bh\\
  A_{base}=b&=\magn{\vec a\times\vec b}\\
  h&=\magn{\vec{c}}*\cos\theta_{cp}
\end{align*}

Where $\theta$ is angle between $\vec{c}$ and a vector perpendicular to both $\vec a$ and $\vec b$, $\vec a\times\vec b$.

\begin{align*}
  (\vec a\times\vec b)\cdot\vec c&=\magn{\vec a\times\vec b}\magn{\vec c}\cos\theta_{cp}\\
  V&=bh\\
  &=\magn{\vec a\times\vec b}\magn{\vec{c}}\cos\theta_{cp}\\
  &=(\vec a\times\vec b)\cdot\vec c\\
\end{align*}

\subsection*{Dot product cosine thing}

Justification for dot product cosine, using law of cosines (given the following):

For the latter statement, imagine law of cosines where $\vec a$ and $vec b$ are the legs, and $\vec c$ is the hypotenuse.  Invert $\vec a$ in your head - notice how $-\vec a+\vec b=\vec c$.

So, prove

\begin{align*}
  \magn{\vec{a}\cdot\vec{b}}=\magn{\vec{a}}*\magn{\vec{b}}*\cos\theta
\end{align*}

given

\begin{align*}
  \vec v=\magn{\vec v}^2\\
  \vec b-\vec a=\vec c\\
  \magn{\vec a}^2+\magn{\vec b}^2-2\magn{\vec a}\magn{\vec b}\cos\theta=\magn{\vec c}^2
\end{align*}


\begin{align*}
  &\vec a\cdot\vec a+\vec b\cdot\vec b-2\magn{\vec a}\magn{\vec b}\cos\theta\\
  &=\vec c\cdot\vec c\\
  &=(\vec b-\vec a)\cdot(\vec b-\vec a)\\
  &=\vec b\cdot\vec b-2\vec b\cdot\vec a+\vec a\cdot\vec a
\end{align*}

\begin{align*}
  -2\magn{\vec a}\magn{\vec b}\cos\theta&=-2\vec a\cdot\vec b\\
  \magn{\vec a}\magn{\vec b}\cos\theta&=\vec a\cdot\vec b
\end{align*}


\end{document}
