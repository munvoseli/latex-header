
\documentclass{article}
\usepackage{geometry}
\geometry{a4paper, portrait, margin=2cm}
\usepackage{amsmath}
\usepackage{mathtools}
\usepackage{graphicx}
\graphicspath{ {./} }
\usepackage[framemethod=TikZ]{mdframed}
\usepackage{lipsum}

\definecolor{boxbg}{HTML}{FFFEB7}
\definecolor{bonkle}{HTML}{FF1500}

\DeclareMathOperator{\sgn}{sign}
\DeclarePairedDelimiter\abs{\lvert}{\rvert}
\DeclarePairedDelimiter\magn{\lvert\lvert}{\rvert\rvert}
\DeclarePairedDelimiter\ceil{\lceil}{\rceil}
\DeclarePairedDelimiter\floor{\lfloor}{\rfloor}
\DeclarePairedDelimiter\vecnum{\langle}{\rangle}

\DeclareRobustCommand{\eel}[1]{\begin{align*}{#1}\end{align*}}
\newmdenv[roundcorner=8pt,align=center,linecolor=boxbg,
  backgroundcolor=boxbg,linewidth=4,outerlinewidth=2,outerlinecolor=bonkle]{stonk}

\begin{document}

On Debian, in addition to the normal recommended package, you'll probably also have to install things like texlive-latex-extra or texlive-science to get all the necessessary things.

\begin{align*}
  \vec a=\vecnum{0,5,3}\\
  \sgn 5+\abs{3-5}\\
  \magn{\vecnum{2,5,14}}=15\\
  \floor{5.5}=\ceil{5}=5\\
  \ceil{5.5}=\floor{6.7}=6
\end{align*}

\begin{align*}
43x^2+27x-5&=0\\
-43*20x^2-2*27*2*5+5*5*2*2&=0\\
-43*20x^2-2*27*10+10^2&=0\\
27^2x^2-2*27*10x+10^2&=(27^2+43*20)x^2\\
(27x-10)^2&=(27^2+43*20)x^2\\
(27-10/x)^2&=27^2+43*20\\
27-\frac{10}x&=\pm\sqrt{27^2+43*20}\\
27\pm\sqrt{27^2+43*20}&=\frac{10}x\\
\frac{10}{27\pm\sqrt{27^2+43*20}}&=x
\end{align*}

And now, some conjugate magic.  It works out if you do both + and - in separate paths tho

\begin{align*}
x&=\frac{10(27\pm\sqrt{27^2+43*20})}{27^2-(27^2+43*20)}\\
&=\frac{10(27\pm\sqrt{27^2+43*20})}{-43*20}\\
&=\frac{-10(27\pm\sqrt{27^2+43*20})}{43*20}\\
&=\frac{-(27\pm\sqrt{27^2+43*20})}{43*2}\\
&=\frac{-27\pm\sqrt{27^2+43*20}}{43*2}\\
&=\frac{-27\pm\sqrt{680+49+860}}{86}\\
&=\frac{-27\pm\sqrt{729+860}}{86}\\
x&=\frac{-27\pm\sqrt{1589}}{86}\\
\end{align*}

\begin{stonk}
hi
\end{stonk}

\section*{Vector product things}

section 12.4, table 11

Cross product produces a vector that is perpendicular to the input vectors.  Magnitude of the cross product is the area of the parallelogram made by the two vectors.

\begin{stonk}
  Cross product works only with 3 dimensional vectors (for now, at least?).
  
  Important equations:
  \begin{align*}
    \magn{\vec a}^2=\vec a\cdot\vec a\\
    \vec{a}\cdot\vec{b}=\magn{\vec{a}}*\magn{\vec{b}}*\cos\theta\\
    \magn{\vec{a}\times\vec{b}}=\magn{\vec{a}}*\magn{\vec{b}}*\sin\theta\\
    A_{2-parallelogram}=\magn{\vec a\times\vec b}\\
    A_{3-parallelogram}=(\vec a\times\vec b)\cdot\vec c\\
    \vec a\perp\vec a\times\vec b\perp\vec b
  \end{align*}
  Fun facts as a result:
  \begin{align*}
    \vec{a}\cdot\vec{b}=0&\Leftrightarrow\vec a\perp\vec b\\
    \magn{\vec{a}\times\vec{b}}=0&\Leftrightarrow\vec a\parallel\vec b\\
  \end{align*}
\end{stonk}


\subsection*{Volume of 3-parallelogram}

\begin{align*}
  V&=bh\\
  A_{base}=b&=\magn{\vec a\times\vec b}\\
  h&=\magn{\vec{c}}*\cos\theta_{cp}
\end{align*}

Where $\theta$ is angle between $\vec{c}$ and a vector perpendicular to both $\vec a$ and $\vec b$, $\vec a\times\vec b$.

\begin{align*}
  (\vec a\times\vec b)\cdot\vec c&=\magn{\vec a\times\vec b}\magn{\vec c}\cos\theta_{cp}\\
  V&=bh\\
  &=\magn{\vec a\times\vec b}\magn{\vec{c}}\cos\theta_{cp}\\
  &=(\vec a\times\vec b)\cdot\vec c\\
\end{align*}

\subsection*{Dot product cosine thing}

Justification for dot product cosine, using law of cosines (given the following):

For the latter statement, imagine law of cosines where $\vec a$ and $vec b$ are the legs, and $\vec c$ is the hypotenuse.  Invert $\vec a$ in your head - notice how $-\vec a+\vec b=\vec c$.

So, prove

\begin{align*}
  \magn{\vec{a}\cdot\vec{b}}=\magn{\vec{a}}*\magn{\vec{b}}*\cos\theta
\end{align*}

given

\begin{align*}
  \vec v=\magn{\vec v}^2\\
  \vec b-\vec a=\vec c\\
  \magn{\vec a}^2+\magn{\vec b}^2-2\magn{\vec a}\magn{\vec b}\cos\theta=\magn{\vec c}^2
\end{align*}


\begin{align*}
  &\vec a\cdot\vec a+\vec b\cdot\vec b-2\magn{\vec a}\magn{\vec b}\cos\theta\\
  &=\vec c\cdot\vec c\\
  &=(\vec b-\vec a)\cdot(\vec b-\vec a)\\
  &=\vec b\cdot\vec b-2\vec b\cdot\vec a+\vec a\cdot\vec a
\end{align*}

\begin{align*}
  -2\magn{\vec a}\magn{\vec b}\cos\theta&=-2\vec a\cdot\vec b\\
  \magn{\vec a}\magn{\vec b}\cos\theta&=\vec a\cdot\vec b
\end{align*}


\end{document}


\begin{document}

On Debian, in addition to the normal recommended package, you'll probably also have to install things like texlive-latex-extra or texlive-science to get all the necessessary things.

\begin{align*}
  \vec a=\vecnum{0,5,3}\\
  \sgn 5+\abs{3-5}\\
  \magn{\vecnum{2,5,14}}=15\\
  \floor{5.5}=\ceil{5}=5\\
  \ceil{5.5}=\floor{6.7}=6
\end{align*}

\begin{align*}
43x^2+27x-5&=0\\
-43*20x^2-2*27*2*5+5*5*2*2&=0\\
-43*20x^2-2*27*10+10^2&=0\\
27^2x^2-2*27*10x+10^2&=(27^2+43*20)x^2\\
(27x-10)^2&=(27^2+43*20)x^2\\
(27-10/x)^2&=27^2+43*20\\
27-\frac{10}x&=\pm\sqrt{27^2+43*20}\\
27\pm\sqrt{27^2+43*20}&=\frac{10}x\\
\frac{10}{27\pm\sqrt{27^2+43*20}}&=x
\end{align*}

And now, some conjugate magic.  It works out if you do both + and - in separate paths tho

\begin{align*}
x&=\frac{10(27\pm\sqrt{27^2+43*20})}{27^2-(27^2+43*20)}\\
&=\frac{10(27\pm\sqrt{27^2+43*20})}{-43*20}\\
&=\frac{-10(27\pm\sqrt{27^2+43*20})}{43*20}\\
&=\frac{-(27\pm\sqrt{27^2+43*20})}{43*2}\\
&=\frac{-27\pm\sqrt{27^2+43*20}}{43*2}\\
&=\frac{-27\pm\sqrt{680+49+860}}{86}\\
&=\frac{-27\pm\sqrt{729+860}}{86}\\
x&=\frac{-27\pm\sqrt{1589}}{86}\\
\end{align*}

\begin{stonk}
hi
\end{stonk}

\section*{Vector product things}

section 12.4, table 11

Cross product produces a vector that is perpendicular to the input vectors.  Magnitude of the cross product is the area of the parallelogram made by the two vectors.

\begin{stonk}
  Cross product works only with 3 dimensional vectors (for now, at least?).
  
  Important equations:
  \begin{align*}
    \magn{\vec a}^2=\vec a\cdot\vec a\\
    \vec{a}\cdot\vec{b}=\magn{\vec{a}}*\magn{\vec{b}}*\cos\theta\\
    \magn{\vec{a}\times\vec{b}}=\magn{\vec{a}}*\magn{\vec{b}}*\sin\theta\\
    A_{2-parallelogram}=\magn{\vec a\times\vec b}\\
    A_{3-parallelogram}=(\vec a\times\vec b)\cdot\vec c\\
    \vec a\perp\vec a\times\vec b\perp\vec b
  \end{align*}
  Fun facts as a result:
  \begin{align*}
    \vec{a}\cdot\vec{b}=0&\Leftrightarrow\vec a\perp\vec b\\
    \magn{\vec{a}\times\vec{b}}=0&\Leftrightarrow\vec a\parallel\vec b\\
  \end{align*}
\end{stonk}

\subsection*{Area of 2-parallelogram}

\begin{align*}
  A&=bh\\
  b&=\magn{\vec a}\\
  h&=\magn{\vec b}\sin\theta_{ab}\\
  A&=\magn{\vec a}\magn{\vec b}\sin\theta_{ab}\\
  A&=\magn{\vec a\times\vec b}
\end{align*}

\subsection*{Volume of 3-parallelogram}

\begin{align*}
  V&=bh\\
  A_{base}=b&=\magn{\vec a\times\vec b}\\
  h&=\magn{\vec{c}}*\cos\theta_{cp}
\end{align*}

Where $\theta_{cp}$ is angle between $\vec{c}$ and a vector perpendicular to both $\vec a$ and $\vec b$, $\vec a\times\vec b$.

\begin{align*}
  (\vec a\times\vec b)\cdot\vec c&=\magn{\vec a\times\vec b}\magn{\vec c}\cos\theta_{cp}\\
  V&=bh\\
  &=\magn{\vec a\times\vec b}\magn{\vec{c}}\cos\theta_{cp}\\
  V&=(\vec a\times\vec b)\cdot\vec c\\
\end{align*}

\subsection*{Dot product cosine thing}

Justification for dot product cosine, using law of cosines (given the following):

For the latter statement, imagine law of cosines where $\vec a$ and $vec b$ are the legs, and $\vec c$ is the hypotenuse.  Invert $\vec a$ in your head - notice how $-\vec a+\vec b=\vec c$.

So, prove

\begin{align*}
  \vec{a}\cdot\vec{b}=\magn{\vec{a}}*\magn{\vec{b}}*\cos\theta
\end{align*}

given

\begin{align*}
  \vec v=\magn{\vec v}^2\\
  \vec b-\vec a=\vec c\\
  \magn{\vec a}^2+\magn{\vec b}^2-2\magn{\vec a}\magn{\vec b}\cos\theta=\magn{\vec c}^2
\end{align*}


\begin{align*}
  &\vec a\cdot\vec a+\vec b\cdot\vec b-2\magn{\vec a}\magn{\vec b}\cos\theta\\
  &=\vec c\cdot\vec c\\
  &=(\vec b-\vec a)\cdot(\vec b-\vec a)\\
  &=\vec b\cdot\vec b-2\vec b\cdot\vec a+\vec a\cdot\vec a
\end{align*}

\begin{align*}
  -2\magn{\vec a}\magn{\vec b}\cos\theta&=-2\vec a\cdot\vec b\\
  \magn{\vec a}\magn{\vec b}\cos\theta&=\vec a\cdot\vec b
\end{align*}

\section*{191 physics equations}

\begin{stonk}
  \begin{align*}
    s_f&=s_i+\int^{t_f}_{t_i}v dt
  \end{align*}
  For constant acceleration:
  \begin{align*}
    v_f&=v_i+a\Delta t\\
    s_f&=s_i+v_i\Delta t+\frac{1}{2}a(\Delta t)^2\\
    v_f^2&=v_i^2+2a\Delta s
  \end{align*}
\end{stonk}

\section*{191 vectors and components}

$\vec A_x$ is a component vector of $\vec A$.  Component vectors are useless, but vector components are not.

$A_x$ and $A_y$ are components of vector $\vec A$. $\vec A = A_x\hat i + A_y\hat j$

\section*{191 word problems}

With the ski problem, he's still on the slope at 15 m/s.  If he were not on the slope, the problem would have more explicitly said something about it.

\section*{252 equations of planes}

normal = perpendicular

A plane in $\mathbb{R}^3$ is uniquely determined by a normal vector and a point in the plane.

\begin{stonk}
  The normal equation of the plane:
  \begin{align*}
    a(x-x_0)+b(y-y_0)+c(z-z_0)&=0\\
    \vec n\cdot(\vec w-\vec p_0)&=0
  \end{align*}
  $\vec w$ is in the plane iff the equation is satisfied.  That plane contains the point $\vec p_0$ and is normal to $\vec n$.
\end{stonk}

To write the equation of a plane given 3 points in the normal equation format, you need a point and a perpendicular vector.  You can generate that perpendicular vector by doing a cross product.

\section*{191 2021-09-03 2d kinematics}

$\Delta r_x$ and $\Delta r_y$ (have the ability to) change simultaneously.

\begin{align*}
  \vec v&=\frac{d\vec r}{dt}\\
  \vec a&=\frac{d\vec v}{dt}\\
  \vec b&=b_x\hat i+b_y\hat j
\end{align*}

Trajectory plots have $x$ and $y$ (space) as axes, not $t$ (time).

The magnitude of the average velocity is the magnitude of the net displacement over $\Delta t$.

To take the derivative of a vector, you can take the derivative of the components.

\section*{252 2021-09-03}

Point $\vec x$ belongs to a line iff a $t$ exists such that $\vec x=t\vec a+\vec b$.

Two lines intersect iff, when you set the equations equal to each other, there exist a value for $s$ and $t$ such that the equation is true (the lines would have the same point).

If they do not intersect, they can either be skew or parallel.

If you have a line ($\vec x=t\vec a+\vec b$) and a plane ($ax + bx + cz = d$), solve for $t$ first to make it easy.

\section*{191 projectile motion}

The projectile starts with initial $x_0$, $y_0$, $v_x_0$, and $v_y_0$, travels with $\vec a$ (whose x component is 0), and ends sometime.

$h$ (height) is the maximum $y$ value.  The range is the horizontal distance between $x_0$ and $x_f$.

\section*{252 vector functions, arc length}

A vector function is a function which maps $\mathbb{R}$ to $\mathbb{R}^3$.

\begin{align*}
  \lim_{t\to u}\vec{f}(t)&=\langle \lim_{t\to u}f_1(t), ...\rangle\\
  \dvec f(t)&=\langle f_1'(t),...\rangle\\
  \int\vec f(t)dt&=\langle\int f_1(t)dt,...\rangle\\
  (ex)&=\langle t+C_1,\frac{t^2}{2}+C_2,...\rangle\\
  &=(t\hat i+\frac{t^2}{2}\hat j+...)+(C_1\hat i+C_2\hat j+...)
\end{align*}


\begin{stonk}
Pretend that the speed of a vector function is the magnitude of its derivative.  You can take the integral of its speed in order to get its length.

\begin{align*}
  L=\int_a^b\sqrt{r_1'^2(t)+r_2'^2(t)+...}
\end{align*}

Where $L$ is the length of the curve.
\end{stonk}

\newpage
\section*{252 computing curvature of vector functions}

\begin{align*}
  s(t_0)=\int_a^{t_0}\magn{\dvec r(t)}dt
\end{align*}

Solve for $t_0$ for the length of curve between two points on curve.

\begin{align*}
  \kappa (s)=\magn{\frac{d\vec T(s)}{ds}}
\end{align*}

Where $\vec T$ is the unit vector of the derivative.

$\kappa$ represents how curved a curve is.

Here are easier formulas.  Use the last one.

\begin{align*}
  \kappa (t)&=\frac{\magn{\dvec T(t)}}{\magn{\dvec r(t)}}\\
  \kappa (t)&=\frac{\magn{\dvec r(t)\times\ddvec r(t)}}{\magn{\xvec[.]{r}(t)}^3}
\end{align*}

\begin{align*}
  \vec T\circ t&=\frac{\dvec r}{\magn{\dvec r}}\\
  \vec N\circ t&=\frac{\dvec T}{\magn{\dvec T}}\\
\end{align*}

\section*{252 chapter 14}

$f:\mathbb{R}^n\to\mathbb{R}$

\section*{191 circular motion}

angular velocity $\omega$. counterclockwise is positive, clockwise is negative

arc length $s=r\theta$

linear speed $v = \frac{ds}{dt}=r\frac{d\theta}{dt}$

acceleration $a=\frac{v^2}r$

period $T=\frac{2\pi}{\omega}$

\begin{align*}
  \omega_{avg}=\frac{\Delta\theta}{\Delta t}
\end{align*}

\section*{252 limits}

\begin{align}
  \lim_{x\to a}g(x)\\
  \lim_{x\to a^+}g(x)\\
  \lim_{x\to a^-}g(x)
\end{align}

(1) exists if (2) and (3) exist and are equal to each other.

But, in $\R^2$, there are many many different ways to approach a different point.  The limit exists in $\R^2$ if the limit is the same along every possible path.

$\lim_{(x,y)\to(0,0)}\frac{2xy}{x^2+3y^2}$ DNE (does not exist) because the limit along the x-axis, which is 0, is different from the limit along the line y=x, which is $\frac 1 2$.

How to prove that the limit does exist?  epsilon delta definition of limit

\begin{stonk}
  Theorem: Polynomials in (x, y) are continuous.
\end{stonk}

A function can be undefined, even when the limit is defined.

\begin{align*}
  f&=\frac x x\\
  f(0)&=DNE\\
  lim_{x\to0}f(x)&=1
\end{align*}

Your path to the point must be defined.


\end{document}
